\documentclass[12pt]{article}

\usepackage{amssymb, amsmath, amsfonts}
\usepackage{amsthm}
\usepackage{moreverb}
\usepackage{graphicx}
\usepackage{enumerate}
\usepackage{graphics}
\usepackage{listings}
\usepackage{color}
\usepackage{array}
\usepackage{float}
\usepackage{hyperref}
\usepackage{textcomp}
\usepackage{caption}
\usepackage{alltt}
\usepackage{mathtools}
\usepackage[T1]{fontenc}
\usepackage{fullpage}
\usepackage{tikz}
\usepackage[utf8]{inputenc}
\newcommand{\suchthat}{\, \mid \,}
\usepackage{fancyvrb}
\allowdisplaybreaks
\def\arraystretch{1.7}%  1 is the default, change whatever you need

\definecolor{codegreen}{rgb}{0,0.6,0}
\definecolor{codegray}{rgb}{0.5,0.5,0.5}
\definecolor{codepurple}{rgb}{0.58,0,0.82}
\definecolor{backcolour}{rgb}{0.95,0.95,0.92}
 
\lstdefinestyle{mystyle}{
    backgroundcolor=\color{backcolour},   
    commentstyle=\color{codegreen},
    identifierstyle=\color{blue},
    keywordstyle=\color{magenta},
    numberstyle=\tiny\color{codegray},
    stringstyle=\color{codepurple},
    basicstyle=\footnotesize\ttfamily,
    breakatwhitespace=false,         
    breaklines=true,                 
    captionpos=b,                    
    keepspaces=true,                 
    numbers=left,                    
    numbersep=5pt,                  
    showspaces=false,                
    showstringspaces=false,
    showtabs=false,                  
    tabsize=2
}
 
\lstset{style=mystyle}

\begin{document}

{\bf MATH 481A \hfill Numerical Analysis \ \ \ \ \ \hfill Spring 2015}

\title{\bf Lab \# 4 Solutions}
\author{\bf Sam Fleischer}
\date{\bf Thurs. Apr. 14, 2015}

{\let\newpage\relax\maketitle}
\maketitle
\tableofcontents
\pagebreak







































\section*{Problem 1}
\addcontentsline{toc}{section}{Problem 1}
{\it Write a function called {\tt trapzoid(f, a, b, n)} that implements the composite trapezoidal rule.  Your function should take as input the name of the function to integrate, {\tt f}, the endpoints of the interval of integration, {\tt a} and {\tt b}, and the number of points {\tt n} of subintervals to divide the integral of integration into.  Use the function to approximate the following integrals with values of {\tt n} $= 10$, $20$, $50$, $100$, and $200$.  In each case indicate the error.  Also, make sure your implementation of the trapezoidal rule does not evaluate the function being integrated more than once at each $x$ value.}
\begin{enumerate}[\ \ (a)\ \ ]
    \item $\displaystyle\int_0^\pi \sin{x}\ dx$ \\\vskip.1cm
    The exact value is
    \begin{align*}
        (-\cos{x})\Big|_{0}^{\pi} = (-\cos(\pi)) - (-\cos(0)) = 1 - (-1) = 2
    \end{align*}
    The error is bounded by $|E_T| \leq \displaystyle\frac{K(b-a)^3}{12n^2}$, where $a = 0$, $b = \pi$, and \\$K = \left|\max\limits_{x\in(0, \pi)}\displaystyle\frac{d^2}{dx^2}\sin{x}\right| = 1$.  So, $E_T \leq \dfrac{\pi^3}{12n^2}$
    \begin{table}[H]
        \begin{tabular}{||c|c|c|c||} \hline\hline
            $n$ & $T_n$ & {\bf Error Bound} & {\bf Actual Error} \\ \hline
            $10$ & $1.9835235375$ & $\frac{\pi^3}{12(10)^2} = 0.025838564$ & $0.01647646250000001$ \\ \hline
            $20$ & $1.9958859727$ & $\frac{\pi^3}{12(20)^2} = 0.006459641$ & $0.004114027299999989$ \\ \hline
            $50$ & $1.9993419831$ & $\frac{\pi^3}{12(50)^2} = 0.001033543$ & $0.0006580168999998914$ \\ \hline
            $100$ & $1.9998355039$ & $\frac{\pi^3}{12(100)^2} = 0.000258386$ & $0.00016449610000002224$ \\ \hline
            $200$ & $1.9999588765$ & $\frac{\pi^3}{12(200)^2} = 6.459640975 \times 10^{-5}$ & $4.112349999996212 \times 10^{-05}$ \\ \hline \hline
        \end{tabular}
    \end{table}
    \item $\displaystyle\int_0^\pi \cos{x}\ dx$ \\\vskip.1cm
    The exact value is
    \begin{align*}
        (\sin{x})\Big|_{0}^{\pi} = (\sin(\pi)) - (\sin(0)) = 0 - 0 = 0
    \end{align*}
    The error is bounded by $|E_T| \leq \displaystyle\frac{K(b-a)^3}{12n^2}$, where $a = 0$, $b = \pi$, and \\$K = \left|\max\limits_{x\in(0, \pi)}\displaystyle\frac{d^2}{dx^2}\cos{x}\right| = 1$.  So, $E_T \leq \dfrac{\pi^3}{12n^2}$
    \begin{table}[H]
        \begin{tabular}{||c|c|c|c||} \hline\hline
            $n$ & $T_n$ & {\bf Error Bound} & {\bf Actual Error} \\ \hline
            $10$ & $3.885781\times 10^{-16}$ & $\frac{\pi^3}{12(10)^2} = 0.025838564$ & $3.885781\times 10^{-16}$ \\ \hline
            $20$ & $1.665335\times 10^{-16}$ & $\frac{\pi^3}{12(20)^2} = 0.006459641$ & $1.665335\times 10^{-16}$ \\ \hline
            $50$ & $-2.775558\times 10^{-17}$ & $\frac{\pi^3}{12(50)^2} = 0.001033543$ & $2.775558\times 10^{-17}$ \\ \hline
            $100$ & $-4.093947\times 10^{-16}$ & $\frac{\pi^3}{12(100)^2} = 0.000258386$ & $4.093947\times 10^{-16}$ \\ \hline
            $200$ & $6.591949\times 10^{-17}$ & $\frac{\pi^3}{12(200)^2} = 6.459640975 \times 10^{-5}$ & $6.591949\times 10^{-17}$ \\ \hline \hline
        \end{tabular}
    \end{table}
    \item $\displaystyle\int_0^{\pi/2} \sin{x}\ dx$ \\\vskip.1cm
    The exact value is
    \begin{align*}
        (-\cos{x})\Big|_{0}^{\frac{\pi}{2}} = (-\cos(\textstyle\frac{\pi}{2})) - (-\cos(0)) = 0 - (-1) = 1
    \end{align*}
    The error is bounded by $|E_T| \leq \displaystyle\frac{K(b-a)^3}{12n^2}$, where $a = 0$, $b = \frac{\pi}{2}$, and \\$K = \left|\max\limits_{x\in(0, \frac{\pi}{2})}\displaystyle\frac{d^2}{dx^2}\sin{x}\right| = 1$.  So, $E_T \leq \dfrac{\pi^3}{96n^2}$
    \begin{table}[H]
        \begin{tabular}{||c|c|c|c||} \hline\hline
            $n$ & $T_n$ & {\bf Error Bound} & {\bf Actual Error} \\ \hline
            $10$ & $0.9979430$ & $\frac{\pi^3}{96(10)^2} = 0.00322982$ & $0.002057$ \\ \hline
            $20$ & $0.9994859$ & $\frac{\pi^3}{96(20)^2} = 0.000807455$ & $0.0005141$ \\ \hline
            $50$ & $0.9999178$ & $\frac{\pi^3}{96(50)^2} = 0.000129193$ & $8.220000 \times 10^{-5}$ \\ \hline
            $100$ & $0.9999794$ & $\frac{\pi^3}{96(100)^2} = 3.229820488 \times 10^{-5}$ & $2.060000 \times 10^{-5}$ \\ \hline
            $200$ & $0.9999949$ & $\frac{\pi^3}{96(200)^2} = 8.074551219 \times 10^{-6}$ & $5.100000 \times 10^{-6}$ \\ \hline \hline
        \end{tabular}
    \end{table}
    \item $\displaystyle\int_0^{\ln{3}} e^{x}\ dx$ \\\vskip.1cm
    The exact value is
    \begin{align*}
        (e^{x})\Big|_{0}^{\ln{3}} = (e^{\ln{3}}) - (e^{0}) = 3 - 1 = 2
    \end{align*}
    The error is bounded by $|E_T| \leq \displaystyle\frac{K(b-a)^3}{12n^2}$, where $a = 0$, $b = \ln{3}$, and \\\vskip.01cm$K = \left|\max\limits_{x\in(0, \ln{3})}\displaystyle\frac{d^2}{dx^2}e^{x}\right| = 3$.  So, $E_T \leq \dfrac{(\ln{3})^3}{4n^2}$
    \begin{table}[H]
        \begin{tabular}{||c|c|c|c||} \hline\hline
            $n$ & $T_n$ & {\bf Error Bound} & {\bf Actual Error} \\ \hline
            $10$ & $2.0020111771$ & $\frac{(\ln{3})^3}{4(10)^2} = 0.003314922$ & $0.0020111771$ \\ \hline
            $20$ & $2.0005028701$ & $\frac{(\ln{3})^3}{4(20)^2} = 0.000828731$ & $0.0005028701$ \\ \hline
            $50$ & $2.0000804626$ & $\frac{(\ln{3})^3}{4(50)^2} = 0.000132597$ & $8.04626 \times 10^{-5}$ \\ \hline
            $100$ & $2.0000201158$ & $\frac{(\ln{3})^3}{4(100)^2} = 3.3149224 \times 10^{-5}$ & $2.01158 \times 10^{-5}$ \\ \hline
            $200$ & $2.0000050290$ & $\frac{(\ln{3})^3}{4(200)^2} = 8.287306001 \times 10^{-6}$ & $5.0290 \times 10^{-6}$ \\ \hline \hline
        \end{tabular}
    \end{table}
\end{enumerate}
The following Python code was used to generate the above approximations:
\begin{lstlisting}[language=Python, caption=Problem 1 source code]
from __future__ import division
from math import sin, cos, pi, exp, log

def irange(start, stop, step):
    r = start
    while r <= stop:
        yield r
        r += step

def trap_area(b_1, b_2, h):
    return (1/2)*(b_1 + b_2)*h

def comp_trap_rule(f, a, b, n):
    h = (b-a)/n

    y_values = [f(a + i*h) for i in irange(0, n, 1)]

    approx = 0
    for i in xrange(0, len(y_values) - 1):
        approx += trap_area(y_values[i], y_values[i+1], h)
    return approx

def problem_1():
    for n in [10, 20, 50, 100, 200]:
        print "\n    n = %d" % n
        a = comp_trap_rule(sin, 0, pi, n)
        b = comp_trap_rule(cos, 0, pi, n)
        c = comp_trap_rule(sin, 0, (pi/2), n)
        d = comp_trap_rule(exp, 0, log(3), n)
        print "        a = %.10e" % a
        print "        b = %.10e" % b
        print "        c = %.10e" % c
        print "        d = %.10e" % d

problem_1()
\end{lstlisting}
The following is the output of the above Python code:
\begin{Verbatim}[fontfamily=courier, numbers=left, numbersep=2pt, fontsize=\small]
    n = 10
        a = 1.9835235375e+00
        b = 3.8857805862e-16
        c = 9.9794298635e-01
        d = 2.0020111771e+00

    n = 20
        a = 1.9958859727e+00
        b = 1.6653345369e-16
        c = 9.9948590525e-01
        d = 2.0005028701e+00

    n = 50
        a = 1.9993419831e+00
        b = -2.7755575616e-17
        c = 9.9991775194e-01
        d = 2.0000804626e+00

    n = 100
        a = 1.9998355039e+00
        b = -4.0939474033e-16
        c = 9.9997943824e-01
        d = 2.0000201158e+00

    n = 200
        a = 1.9999588765e+00
        b = 6.5919492087e-17
        c = 9.9999485958e-01
        d = 2.0000050290e+00
\end{Verbatim}











































\pagebreak
\section*{Problem 2}
\addcontentsline{toc}{section}{Problem 2}
{\it Wtie a function {\tt midpoint(f, a, b, n)} that implements the composite midpoint rule.  Your function should take as input the name of the function to integrate, {\tt f}, the endpoints of the interval of integration, {\tt a} and {\tt b}, and the number of points {\tt n} of subintervals to divide the integral of integration into.  Use the function to approximate the same integrals as in {\bf Problem 1} with values of {\tt n} $= 10$, $20$, amd $50$.  In each case indicate the error.}
\begin{enumerate}[\ \ (a)\ \ ]
    \item $\displaystyle\int_0^\pi \sin{x}\ dx$ \\\vskip.1cm
    The error is bounded by $|E_M| \leq \displaystyle\frac{K(b-a)^3}{24n^2}$, where $a = 0$, $b = \pi$, and \\$K = \left|\max\limits_{x\in(0, \pi)}\displaystyle\frac{d^2}{dx^2}\sin{x}\right| = 1$.  So, $E_T \leq \dfrac{\pi^3}{24n^2}$
    \begin{table}[H]
        \begin{tabular}{||c|c|c|c||} \hline\hline
            $n$ & $M_n$ & {\bf Error Bound} & {\bf Actual Error} \\ \hline
            $10$ & $2.0082484079$ & $\frac{\pi^3}{24(10)^2} = 0.012919282$ & $0.0082484079$ \\ \hline
            $20$ & $2.0020576483$ & $\frac{\pi^3}{24(20)^2} = 0.00322982$ & $0.0020576483$ \\ \hline
            $50$ & $2.0003290247$ & $\frac{\pi^3}{24(50)^2} = 0.000516771$ & $0.0003290247$ \\ \hline \hline
        \end{tabular}
    \end{table}
    \item $\displaystyle\int_0^\pi \cos{x}\ dx$ \\\vskip.1cm
    The error is bounded by $|E_M| \leq \displaystyle\frac{K(b-a)^3}{24n^2}$, where $a = 0$, $b = \pi$, and \\$K = \left|\max\limits_{x\in(0, \pi)}\displaystyle\frac{d^2}{dx^2}\cos{x}\right| = 1$.  So, $E_T \leq \dfrac{\pi^3}{24n^2}$
    \begin{table}[H]
        \begin{tabular}{||c|c|c|c||} \hline\hline
            $n$ & $M_n$ & {\bf Error Bound} & {\bf Actual Error} \\ \hline
            $10$ & $-5.5511151231 \times 10^{-17}$ & $\frac{\pi^3}{24(10)^2} = 0.012919282$ & $5.5511151231 \times 10^{-17}$ \\ \hline
            $20$ & $-1.3877787808 \times 10^{-16}$ & $\frac{\pi^3}{24(20)^2} = 0.00322982$ & $1.3877787808 \times 10^{-16}$ \\ \hline
            $50$ & $3.0531133177 \times 10^{-16}$ & $\frac{\pi^3}{24(50)^2} = 0.000516771$ & $3.0531133177 \times 10^{-16}$ \\ \hline \hline
        \end{tabular}
    \end{table}
    \item $\displaystyle\int_0^{\pi/2} \sin{x}\ dx$ \\\vskip.1cm
    The error is bounded by $|E_M| \leq \displaystyle\frac{K(b-a)^3}{24n^2}$, where $a = 0$, $b = \frac{\pi}{2}$, and \\$K = \left|\max\limits_{x\in(0, \frac{\pi}{2})}\displaystyle\frac{d^2}{dx^2}\sin{x}\right| = 1$.  So, $E_T \leq \dfrac{\pi^3}{192n^2}$
    \begin{table}[H]
        \begin{tabular}{||c|c|c|c||} \hline\hline
            $n$ & $M_n$ & {\bf Error Bound} & {\bf Actual Error} \\ \hline
            $10$ & $1.0010288241$ & $\frac{\pi^3}{192(10)^2} = 0.00161491$ & $0.0010288241$ \\ \hline
            $20$ & $1.0002570672$ & $\frac{\pi^3}{192(20)^2} = 0.000403728$ & $0.0002570672$ \\ \hline
            $50$ & $1.0000411245$ & $\frac{\pi^3}{192(50)^2} = 6.4596 \times 10^{-5}$ & $4.11245 \times 10^{-5}$ \\ \hline \hline
        \end{tabular}
    \end{table}
    \item $\displaystyle\int_0^{\ln{3}} e^{x}\ dx$ \\\vskip.1cm
    The error is bounded by $|E_M| \leq \displaystyle\frac{K(b-a)^3}{24n^2}$, where $a = 0$, $b = \ln{3}$, and \\\vskip.01cm$K = \left|\max\limits_{x\in(0, \ln{3})}\displaystyle\frac{d^2}{dx^2}e^{x}\right| = 3$.  So, $E_T \leq \dfrac{(\ln{3})^3}{8n^2}$
    \begin{table}[H]
        \begin{tabular}{||c|c|c|c||} \hline\hline
            $n$ & $M_n$ & {\bf Error Bound} & {\bf Actual Error} \\ \hline
            $10$ & $1.9989945632$ & $\frac{(\ln{3})^3}{8(10)^2} = 0.001657461$ & $0.0010054368$ \\ \hline
            $20$ & $1.9997485744$ & $\frac{(\ln{3})^3}{8(20)^2} = 0.000414365$ & $0.0002514256$ \\ \hline
            $50$ & $1.9999597689$ & $\frac{(\ln{3})^3}{8(50)^2} = 6.6298 \times 10^{-5}$ & $4.02311 \times 10^{-5}$ \\ \hline \hline
        \end{tabular}
    \end{table}
\end{enumerate}
The following Python code was used to generate the above approximations:
\begin{lstlisting}[language=Python, caption=Problem 2 source code]
from __future__ import division
from math import sin, cos, pi, exp, log

def rect_area(b, h):
    return b*h

def comp_mdpt_rule(f, a, b, n):
    h = (b-a)/n

    approx = 0
    for i in xrange(0, n):
        approx += rect_area(h, f(a + i*h + (1/2)*h))
    return approx

def problem_2():
    for n in [10, 20, 50]:
        print "\n    n = %d" % n
        a = comp_mdpt_rule(sin, 0, pi, n)
        b = comp_mdpt_rule(cos, 0, pi, n)
        c = comp_mdpt_rule(sin, 0, (pi/2), n)
        d = comp_mdpt_rule(exp, 0, log(3), n)
        print "        a = %.10e" % a
        print "        b = %.10e" % b
        print "        c = %.10e" % c
        print "        d = %.10e" % d


problem_2()
\end{lstlisting}
The following is the output of the above Python code:
\begin{Verbatim}[fontfamily=courier, numbers=left, numbersep=2pt, fontsize=\small]
    n = 10
        a = 2.0082484079e+00
        b = -5.5511151231e-17
        c = 1.0010288241e+00
        d = 1.9989945632e+00

    n = 20
        a = 2.0020576483e+00
        b = -1.3877787808e-16
        c = 1.0002570672e+00
        d = 1.9997485744e+00

    n = 50
        a = 2.0003290247e+00
        b = 3.0531133177e-16
        c = 1.0000411245e+00
        d = 1.9999597689e+00
\end{Verbatim}








































\pagebreak
\section*{Problem 3}
\addcontentsline{toc}{section}{Problem 3}
{\it Wtie a function {\tt midpoint(f, a, b, n)} that implements the composite midpoint rule.  Your function should take as input the name of the function to integrate, {\tt f}, the endpoints of the interval of integration, {\tt a} and {\tt b}, and the number of points {\tt n} of subintervals to divide the integral of integration into.  Use the function to approximate the same integrals as in {\bf Problem 1} with values of {\tt n} $= 10$, $20$, amd $50$.  In each case indicate the error.}
\begin{enumerate}[\ \ (a)\ \ ]
    \item $\displaystyle\int_0^\pi \sin{x}\ dx$ \\\vskip.1cm
    The error is bounded by $|E_S| \leq \displaystyle\frac{K(b-a)^5}{180n^4}$, where $a = 0$, $b = \pi$, and \\$K = \left|\max\limits_{x\in(0, \pi)}\displaystyle\frac{d^4}{dx^4}\sin{x}\right| = 1$.  So, $E_T \leq \dfrac{\pi^5}{180n^4}$
    \begin{table}[H]
        \begin{tabular}{||c|c|c|c||} \hline\hline
            $n$ & $S_n$ & {\bf Error Bound} & {\bf Actual Error} \\ \hline
            $10$ & $2.0001095173$ & $\frac{\pi^5}{180(10)^4} = 0.000170011$ & $0.0001095173$ \\ \hline
            $20$ & $2.0000067844$ & $\frac{\pi^5}{180(20)^4} = 1.06256835 \times 10^{-5}$ & $6.7844 \times 10^{-6}$ \\ \hline
            $50$ & $2.0000001733$ & $\frac{\pi^5}{180(50)^4} = 2.720174976 \times 10^{-7}$ & $1.733 \times 10^{-7}$ \\ \hline
            $100$ & $2.0000000108$ & $\frac{\pi^5}{180(100)^4} = 1.70010936 \times 10^{-8}$ & $1.08 \times 10^{-8}$ \\ \hline
            $200$ & $2.0000000007$ & $\frac{\pi^5}{180(200)^4} = 1.06256835 \times 10^{-9}$ & $7 \times 10 ^{-10}$ \\ \hline \hline
        \end{tabular}
    \end{table}
    \item $\displaystyle\int_0^\pi \cos{x}\ dx$ \\\vskip.1cm
    The error is bounded by $|E_S| \leq \displaystyle\frac{K(b-a)^5}{180n^4}$, where $a = 0$, $b = \pi$, and \\$K = \left|\max\limits_{x\in(0, \pi)}\displaystyle\frac{d^4}{dx^4}\cos{x}\right| = 1$.  So, $E_S \leq \dfrac{\pi^5}{180n^4}$
    \begin{table}[H]
        \begin{tabular}{||c|c|c|c||} \hline\hline
            $n$ & $S_n$ & {\bf Error Bound} & {\bf Actual Error} \\ \hline
            $10$ & $0.0000000000$ & $\frac{\pi^5}{180(10)^4} = 0.000170011$ & $0.0000000000$ \\ \hline
            $20$ & $3.3306690739 \times 10^{-16}$ & $\frac{\pi^5}{180(20)^4} = 1.06256835 \times 10^{-5}$ & $3.3306690739 \times 10^{-16}$ \\ \hline
            $50$ & $-8.3266726847 \times 10^{-17}$ & $\frac{\pi^5}{180(50)^4} = 2.720174976 \times 10^{-7}$ & $8.3266726847 \times 10^{-17}$ \\ \hline
            $100$ & $-5.2735593670 \times 10^{-16}$ & $\frac{\pi^5}{180(100)^4} = 1.70010936 \times 10^{-8}$ & $5.2735593670 \times 10^{-16}$ \\ \hline
            $200$ & $6.9388939039 \times 10^{-18}$ & $\frac{\pi^5}{180(200)^4} = 1.06256835 \times 10^{-9}$ & $6.9388939039 \times 10^{-18}$ \\ \hline \hline
        \end{tabular}
    \end{table}
    \item $\displaystyle\int_0^{\pi/2} \sin{x}\ dx$ \\\vskip.1cm
    The error is bounded by $|E_S| \leq \displaystyle\frac{K(b-a)^5}{180n^4}$, where $a = 0$, $b = \frac{\pi}{2}$, and \\$K = \left|\max\limits_{x\in(0, \frac{\pi}{2})}\displaystyle\frac{d^4}{dx^4}\sin{x}\right| = 1$.  So, $E_S \leq \dfrac{\pi^5}{5760n^4}$
    \begin{table}[H]
        \begin{tabular}{||c|c|c|c||} \hline\hline
            $n$ & $S_n$ & {\bf Error Bound} & {\bf Actual Error} \\ \hline
            $10$ & $1.000003392220900$ & $\frac{\pi^5}{5760(10)^4} = 5.31284175 \times 10^{-6}$ & $3.392220900 \times 10^{-6}$ \\ \hline
            $20$ & $1.000000211546591$ & $\frac{\pi^5}{5760(20)^4} = 3.320526094 \times 10^{-7}$ & $2.11546591 \times 10^{-7}$ \\ \hline
            $50$ & $1.000000005412252$ & $\frac{\pi^5}{5760(50)^4} = 8.5005468 \times 10^{-9}$ & $5.412252 \times 10^{-9}$ \\ \hline
            $100$ & $1.000000000338236$ & $\frac{\pi^5}{5760(100)^4} = 5.3128417 \times 10^{-10}$ & $3.38236 \times 10^{-10}$ \\ \hline
            $200$ & $1.000000000021139$ & $\frac{\pi^5}{5760(200)^4} = 3.3205261 \times 10^{-11}$ & $2.1139 \times 10^{-11}$ \\ \hline \hline
        \end{tabular}
    \end{table}
    \item $\displaystyle\int_0^{\ln{3}} e^{x}\ dx$ \\\vskip.1cm
    The error is bounded by $|E_S| \leq \displaystyle\frac{K(b-a)^5}{180n^4}$, where $a = 0$, $b = \ln{3}$, and \\\vskip.01cm$K = \left|\max\limits_{x\in(0, \ln{3})}\displaystyle\frac{d^4}{dx^4}e^{x}\right| = 3$.  So, $E_T \leq \dfrac{(\ln{3})^5}{60n^4}$
    \begin{table}[H]
        \begin{tabular}{||c|c|c|c||} \hline\hline
            $n$ & $S_n$ & {\bf Error Bound} & {\bf Actual Error} \\ \hline
            $10$ & $2.000001616261506$ & $\frac{(\ln{3})^5}{60(10)^4} = 2.667294764 \times 10^{-6}$ & $1.616261506 \times 10^{-6}$ \\ \hline
            $20$ & $2.000000101125187$ & $\frac{(\ln{3})^5}{60(20)^4} = 1.667059228 \times 10^{-7}$ & $1.01125187 \times 10^{-7}$ \\ \hline
            $50$ & $2.000000002589586$ & $\frac{(\ln{3})^5}{60(50)^4} = 4.267671623 \times 10^{-9}$ & $2.589586 \times 10^{-9}$ \\ \hline
            $100$ & $2.000000000161856$ & $\frac{(\ln{3})^5}{60(100)^4} = 2.6672948 \times 10^{-10}$ & $1.61856 \times 10^{-10}$ \\ \hline
            $200$ & $2.000000000010117$ & $\frac{(\ln{3})^5}{60(200)^4} = 1.6670592 \times 10^{-11}$ & $1.0117 \times 10^{-11}$ \\ \hline \hline
        \end{tabular}
    \end{table}
\end{enumerate}
The following Python code was used to generate the above approximations:
\begin{lstlisting}[language=Python, caption=Problem 3 source code]
from __future__ import division
from math import sin, cos, pi, exp, log

def rect_area(b, h):
    return b*h

def trap_area(b_1, b_2, h):
    return (1/2)*(b_1 + b_2)*h

def simp_rule(f, a, b):
    M = rect_area(f((a+b)/2), (b-a))
    T = trap_area(f(a), f(b), (b-a))
    return (2*M + T)/3

def comp_simp_rule(f, a, b, n):
    h = (b-a)/n

    approx = 0
    for i in xrange(0, int(n/2)):
        approx += simp_rule(f, a+(2*i*h), a+((2*i+2)*h))
    return approx    

def problem_3():
    for n in [10, 20, 50, 100, 200]:
        print "\n    n = %d" % n
        a = comp_simp_rule(sin, 0, pi, n)
        b = comp_simp_rule(cos, 0, pi, n)
        c = comp_simp_rule(sin, 0, (pi/2), n)
        d = comp_simp_rule(exp, 0, log(3), n)
        print "        a = %.10e" % a
        print "        b = %.10e" % b
        print "        c = %.10e" % c
        print "        d = %.10e" % d

problem_3()
\end{lstlisting}
The following is the output of the above Python code:
\begin{Verbatim}[fontfamily=courier, numbers=left, numbersep=2pt, fontsize=\small]
    n = 10
        a = 2.0001095173e+00
        b = 0.0000000000e+00
        c = 1.000003392220900e+00
        d = 2.000001616261506e+00

    n = 20
        a = 2.0000067844e+00
        b = 3.3306690739e-16
        c = 1.000000211546591e+00
        d = 2.000000101125187e+00

    n = 50
        a = 2.0000001733e+00
        b = -8.3266726847e-17
        c = 1.000000005412252e+00
        d = 2.000000002589586e+00

    n = 100
        a = 2.0000000108e+00
        b = -5.2735593670e-16
        c = 1.000000000338236e+00
        d = 2.000000000161856e+00

    n = 200
        a = 2.0000000007e+00
        b = 6.9388939039e-18
        c = 1.000000000021139e+00
        d = 2.000000000010117e+00
\end{Verbatim}









































\pagebreak
\section*{Problem 4}
\addcontentsline{toc}{section}{Problem 4}
{\it Write a function {\tt Dfwd(f, x, h=1e-6)} that implements the 1st order finite difference approximation of the derivative}
\begin{align*}
    f'(x) \approx \frac{f(x + h) - f(x)}{h}
\end{align*}
{\it Use this formula to approximate the derivative of the following functions at the specified values of $x$ for $h = 0.1$, $0.05$, $0.01$, and $0.001$ and calculate the error and the error bound.  Present your results for each function in a table like the one for the revious problems:}
\begin{enumerate}[\ \ (a)\ \ ]
    \item $f(x) = e^x$ at $x = 0$.  The exact value is
    \begin{align*}
        f'(x) &= e^x \\
        f'(0) &= e^0 = 1
    \end{align*}
    The error is bounded by $|E| \leq \displaystyle\frac{hK}{2}$, where $K = \max\limits_{x\in(0, h)}\left|f''(x)\right|$.  Since $f''(x) = e^x$ is an increasing positive function, $K = e^h$.  So, $E \leq \dfrac{he^h}{2}$
    \begin{table}[H]
        \begin{tabular}{||c|c|c|c||} \hline\hline
            $h$ & $\sim f'(0)$ & {\bf Error Bound} & {\bf Actual Error} \\ \hline
            $0.1$ & $1.0517091808$ & $\frac{0.1e^{0.1}}{2} = 0.0552585$ & $0.0517091808$ \\ \hline
            $0.05$ & $1.0254219275$ & $\frac{0.051e^{0.05}}{2} = 0.026281777$ & $0.0254219275$ \\ \hline
            $0.01$ & $1.0050167084$ & $\frac{0.01e^{0.01}}{2} = 0.00505025$ & $0.0050167084$ \\ \hline
            $0.001$ & $1.0005001667$ & $\frac{0.001e^{0.001}}{2} = 0.000500500$ & $0.0005001667$ \\ \hline \hline
        \end{tabular}
    \end{table}
    \item $f(x) = e^{-2x^2}$ at $x = 0$.  The exact value is
    \begin{align*}
        f'(x) &= -4x\ e^{-2x^2} \\
        f'(0) &= -4(0)\ e^0 = 0
    \end{align*}
    The error is bounded by $|E| \leq \displaystyle\frac{hK}{2}$, where $K = \max\limits_{x\in(0, h)}\left|f''(x)\right|$.  Since $f''(x) = 4e^{-2x^2}(4x^2 - 1)$ is an increasing negative function on $(0, h)$ (for the $h$'s we are considering), $K = |f''(0)| = 4$.  So, $E \leq 2h$
    \begin{table}[H]
        \begin{tabular}{||c|c|c|c||} \hline\hline
            $h$ & $\sim f'(0)$ & {\bf Error Bound} & {\bf Actual Error} \\ \hline
            $0.1$ & $-0.19801326693$ & $2(0.1) = 0.2$ & $0.19801326693$ \\ \hline
            $0.05$ & $-0.099750416146$ & $2(0.05) = 0.1$ & $0.099750416146$ \\ \hline
            $0.01$ & $-0.019998000133$ & $2(0.01) = 0.02$ & $0.019998000133$ \\ \hline
            $0.001$ & $-0.0019999980000$ & $2(0.001) = 0.002$ & $0.0019999980000$ \\ \hline \hline
        \end{tabular}
    \end{table}
    \item $f(x) = \cos{x}$ at $x = 2\pi$.  The exact value is
    \begin{align*}
        f'(x) &= -\sin{x} \\
        f'(0) &= -\sin{0} = 0
    \end{align*}
    The error is bounded by $|E| \leq \displaystyle\frac{hK}{2}$, where $K = \max\limits_{x\in(2\pi, 2\pi + h)}\left|f''(x)\right|$.  Since $f''(x) = -\cos{x}$ is an increasing negative function on $(2\pi, 2\pi + h)$ (for the $h$'s we are considering), $K = |f''(0)| = 1$.  So, $E \leq \displaystyle\frac{h}{2}$
    \begin{table}[H]
        \begin{tabular}{||c|c|c|c||} \hline\hline
            $h$ & $\sim f'(0)$ & {\bf Error Bound} & {\bf Actual Error} \\ \hline
            $0.1$ & $-0.049958347220$ & $\frac{0.1}{2} = 0.05$ & $0.049958347220$ \\ \hline
            $0.05$ & $-0.024994792101$ & $\frac{0.05}{2} = 0.025$ & $0.024994792101$ \\ \hline
            $0.01$ & $-0.0049999583335$ & $\frac{0.01}{2} = 0.005$ & $0.0049999583335$ \\ \hline
            $0.001$ & $-0.00049999995833$ & $\frac{0.001}{2} = 0.0005$ & $0.00049999995833$ \\ \hline \hline
        \end{tabular}
    \end{table}
    \item $f(x) = \ln{x}$ at $x = 1$.  The exact value is
    \begin{align*}
        f'(x) &= \frac{1}{x} \\
        f'(1) &= \frac{1}{1} = 1
    \end{align*}
    The error is bounded by $|E| \leq \displaystyle\frac{hK}{2}$, where $K = \max\limits_{x\in(1, 1 + h)}\left|f''(x)\right|$.  Since $f''(x) = -\dfrac{1}{x^2}$ is an increasing negative function on $(1, 1 + h)$, $K = |f''(1)| = 1$.  So, $E \leq \displaystyle\frac{h}{2}$
    \begin{table}[H]
        \begin{tabular}{||c|c|c|c||} \hline\hline
            $h$ & $\sim f'(0)$ & {\bf Error Bound} & {\bf Actual Error} \\ \hline
            $0.1$ & $0.95310179804$ & $\frac{0.1}{2} = 0.05$ & $0.04689820196$ \\ \hline
            $0.05$ & $0.97580328339$ & $\frac{0.05}{2} = 0.025$ & $0.02419671661$ \\ \hline
            $0.01$ & $0.99503308532$ & $\frac{0.01}{2} = 0.005$ & $0.00496691468$ \\ \hline
            $0.001$ & $0.99950033308$ & $\frac{0.001}{2} = 0.0005$ & $0.00049966692$ \\ \hline \hline
        \end{tabular}
    \end{table}

\end{enumerate}
The following Python code was used to generate the above approximations:
\begin{lstlisting}[language=Python, caption=Problem 3 source code]
from __future__ import division
from math import cos, pi, exp, log

def first_order_deriv_approx(f, x, h=1e-6):
    return (f(x+h) - f(x))/h

def exp_4b(x):
    return exp(-2*(x**2))

def problem_4():
    for h in [0.1, 0.05, 0.01, 0.001]:
        print "\n    h = %.3f" % h
        a = first_order_deriv_approx(exp, 0, h)
        b = first_order_deriv_approx(exp_4b, 0, h)
        c = first_order_deriv_approx(cos, 2*pi, h)
        d = first_order_deriv_approx(log, 1, h)
        print "        a = %.10e" % a
        print "        b = %.10e" % b
        print "        c = %.10e" % c
        print "        d = %.10e" % d

problem_4()
\end{lstlisting}
The following is the output of the above Python code:
\begin{Verbatim}[fontfamily=courier, numbers=left, numbersep=2pt, fontsize=\small]
    h = 0.100
        a = 1.0517091808e+00
        b = -1.9801326693e-01
        c = -4.9958347220e-02
        d = 9.5310179804e-01

    h = 0.050
        a = 1.0254219275e+00
        b = -9.9750416146e-02
        c = -2.4994792101e-02
        d = 9.7580328339e-01

    h = 0.010
        a = 1.0050167084e+00
        b = -1.9998000133e-02
        c = -4.9999583335e-03
        d = 9.9503308532e-01

    h = 0.001
        a = 1.0005001667e+00
        b = -1.9999980000e-03
        c = -4.9999995833e-04
        d = 9.9950033308e-01
\end{Verbatim}





































\pagebreak
\section*{Problem 5}
\addcontentsline{toc}{section}{Problem 5}
{\it Repeat {\bf Problem 4} for the 2nd order centered difference}
\begin{align*}
    f'(x) \approx \frac{f(x + h) - f(x - h)}{2h}
\end{align*}
\begin{enumerate}[\ \ (a)\ \ ]
    \item $f(x) = e^x$ at $x = 0$.
    The error is bounded by $|E| \leq \displaystyle\frac{h^2K}{6}$, where $K = \max\limits_{x\in(-h, h)}\left|f'''(x)\right|$.  Since $f'''(x) = e^x$ is an increasing positive function, $K = e^h$.  So, $E \leq \dfrac{h^2e^h}{6}$
    \begin{table}[H]
        \begin{tabular}{||c|c|c|c||} \hline\hline
            $h$ & $\sim f'(0)$ & {\bf Error Bound} & {\bf Actual Error} \\ \hline
            $0.1$ & $1.001667500198441$ & $\frac{0.1^2e^{0.1}}{6} = 0.00184195$ & $0.001667500198441$ \\ \hline
            $0.05$ & $1.000416718753101$ & $\frac{0.051^2e^{0.05}}{6} = 0.0004380296$ & $0.000416718753101$ \\ \hline
            $0.01$ & $1.000016666749992$ & $\frac{0.01^2e^{0.01}}{6} = 1.6834169 \times 10^{-5}$ & $1.6666749992 \times 10^{-5}$ \\ \hline
            $0.001$ & $1.000000166666681$ & $\frac{0.001^2e^{0.001}}{6} = 1.668334 \times 10^{-7}$ & $1.66666681 \times 10^{-7}$ \\ \hline \hline
        \end{tabular}
    \end{table}
    \item $f(x) = e^{-2x^2}$ at $x = 0$.
    The error is bounded by $|E| \leq \displaystyle\frac{h^2K}{6}$, where $K = \max\limits_{x\in(-h, h)}\left|f'''(x)\right|$.  Since $f'''(x) = -16xe^{-2x^2}(4x^2 - 3)$ is an increasing positive function on $(0, h)$ (for the $h$'s we are considering) and $f'''(x)$ is odd, $K = |f'''(h)|$.  So, $E \leq \dfrac{-8h^3(4h^2 - 3)e^{-2h^2}}{3}$
    \begin{table}[H]
        \begin{tabular}{||c|c|c|c||} \hline\hline
            $h$ & $\sim f'(0)$ & {\bf Error Bound} & {\bf Actual Error} \\ \hline
            $0.1$ & $0.0000000000$ & $\frac{-8(0.1)^3(4(0.1)^2 - 3)e^{-2(0.1)^2}}{3} = 0.00773703486$ & $0.000000000000000$ \\ \hline
            $0.05$ & $0.0000000000$ & $\frac{-8(0.05)^3(4(0.05)^2 - 3)e^{-2(0.05)^2}}{3} = 0.40435377$ & $0.000000000000000$ \\ \hline
            $0.01$ & $0.0000000000$ & $\frac{-8(0.01)^3(4(0.01)^2 - 3)e^{-2(0.01)^2}}{3} = 7.9973337 \times 10^{-6}$ & $0.000000000000000$ \\ \hline
            $0.001$ & $0.0000000000$ & $\frac{-8(0.001)^3(4(0.001)^2 - 3)e^{-2(0.001)^2}}{3} = 7.99997 \times 10^{-9}$ & $0.000000000000000$ \\ \hline \hline
        \end{tabular}
    \end{table}
    \item $f(x) = \cos{x}$ at $x = 2\pi$.
    The error is bounded by $|E| \leq \displaystyle\frac{h^2K}{6}$, where $K = \max\limits_{x\in(2\pi - h, 2\pi + h)}\left|f'''(x)\right|$.  Since $f'''(x) = \sin{x}$ is an increasing negative function on $(2\pi, 2\pi + h)$ (for the $h$'s we are considering) and $f'''(x)$ is odd and $2\pi$-periodic, $K = |f'''(2\pi + h)| = sin(h)$.  So, $E \leq \displaystyle\frac{h^2\sin{h}}{6}$
    \begin{table}[H]
        \begin{tabular}{||c|c|c|c||} \hline\hline
            $h$ & $\sim f'(0)$ & {\bf Error Bound} & {\bf Actual Error} \\ \hline
            $0.1$ & $0.000000000000000$ & $\frac{(0.1)^2\sin{(0.1)}}{6} = 0.000166389$ & $0.000000000000000$ \\ \hline
            $0.05$ & $0.000000000000000$ & $\frac{(0.05)^2\sin{(0.05)}}{6} = 2.082465386 \times 10^{-5}$ & $0.000000000000000$ \\ \hline
            $0.01$ & $0.000000000000000$ & $\frac{(0.01)^2\sin{(0.01)}}{6} = 1.666638889 \times 10^{-7}$ & $0.000000000000000$ \\ \hline
            $0.001$ & $0.000000000000000$ & $\frac{(0.001)^2\sin{(0.001)}}{6} = 1.666666 \times 10^{-10}$ & $0.000000000000000$ \\ \hline \hline
        \end{tabular}
    \end{table}
    \item $f(x) = \ln{x}$ as $x = 1$.
    The error is bounded by $|E| \leq \displaystyle\frac{h^2K}{6}$, where $K = \max\limits_{x\in(1 - h, 1 + h)}\left|f'''(x)\right|$.  Since $f'''(x) = \dfrac{2}{x^3}$ is a decreasing positive function on $(1 - h, 1 + h)$ (for the $h$'s we are considering), $K = |f'''(1 - h)|$.  So, $E \leq \displaystyle\frac{h^2}{3(1 - h)^3}$
    \begin{table}[H]
        \begin{tabular}{||c|c|c|c||} \hline\hline
            $h$ & $\sim f'(0)$ & {\bf Error Bound} & {\bf Actual Error} \\ \hline
            $0.1$ & $1.003353477310756$ & $\frac{(0.1)^2}{3(1 - (0.1))^3} = 0.00457247$ & $0.003353477310756$ \\ \hline
            $0.05$ & $1.000834585569826$ & $\frac{(0.05)^2}{3(1 - (0.05))^3} = 0.00097195898$ & $0.000834585569826$ \\ \hline
            $0.01$ & $1.000033335333477$ & $\frac{(0.01)^2}{3(1 - (0.01))^3} = 3.43536717 \times 10^{-5}$ & $3.3335333477 \times 10^{-5}$ \\ \hline
            $0.001$ & $1.000000333333479$ & $\frac{(0.001)^2}{3(1 - (0.001))^3} = 3.3433534 \times 10^{-7}$ & $3.33333479 \times 10^{-7}$ \\ \hline \hline
        \end{tabular}
    \end{table}

\end{enumerate}
The following Python code was used to generate the above approximations:
\begin{lstlisting}[language=Python, caption=Problem 3 source code]
from __future__ import division
from math import cos, pi, exp, log

def second_order_deriv_approx(f, x, h=1e-6):
    return (f(x+h) - f(x-h))/(2*h)

def exp_4b(x):
    return exp(-2*(x**2))

def problem_5():
    for h in [0.1, 0.05, 0.01, 0.001]:
        print "\n    h = %.3f" % h
        a = second_order_deriv_approx(exp, 0, h)
        b = second_order_deriv_approx(exp_4b, 0, h)
        c = second_order_deriv_approx(cos, 2*pi, h)
        d = second_order_deriv_approx(log, 1, h)
        print "        a = %.15e" % a
        print "        b = %.15e" % b
        print "        c = %.15e" % c
        print "        d = %.15e" % d

problem_5()
\end{lstlisting}
The following is the output of the above Python code:
\begin{Verbatim}[fontfamily=courier, numbers=left, numbersep=2pt, fontsize=\small]
    h = 0.100
        a = 1.001667500198441e+00
        b = 0.000000000000000e+00
        c = 0.000000000000000e+00
        d = 1.003353477310756e+00

    h = 0.050
        a = 1.000416718753101e+00
        b = 0.000000000000000e+00
        c = 0.000000000000000e+00
        d = 1.000834585569826e+00

    h = 0.010
        a = 1.000016666749992e+00
        b = 0.000000000000000e+00
        c = 0.000000000000000e+00
        d = 1.000033335333477e+00

    h = 0.001
        a = 1.000000166666681e+00
        b = 0.000000000000000e+00
        c = 0.000000000000000e+00
        d = 1.000000333333479e+00
\end{Verbatim}









































\pagebreak
\section*{Problem 6}
\addcontentsline{toc}{section}{Problem 6}
{\it A particle moves along a trajectory in the $xy$ plane such that at time $t$ the object has position $(x(t), y(t))$.  The velocity vector of the particle can be approximated by}
\begin{align*}
    v(t) \approx \left(\frac{x(t + \Delta t) - x(t)}{\Delta t^2}, \frac{y(t + \Delta t) - y(t)}{\Delta t^2}\right)
\end{align*}
{\it and its acceleration by}
\begin{align*}
    a(t) \approx \left(\frac{x(t + \Delta t) - 2x(t) + x(t - \Delta t)}{\Delta t^2}, \frac{y(t + \Delta t) - 2y(t) + y(t - \Delta t)}{\Delta t^2}\right)
\end{align*}
{\it Write a function {\tt kinematics(x, y, t, dt=1.0e-6)} that approximates the velocity and acceleration of a particle with position vector $(\cos{2\pi t}, \sin{2\pi t})$ for {\tt t} $ = 0$, $\frac{1}{4}$, $\frac{1}{2}$, $\frac{3}{4}$, and {\tt dt} $ = 0.1$, $0.05$, and $0.01$.  Calculate the error in the approximations.  Display your results in a table.} \\

\noindent The actual solutions are:
\begin{align*}
    v(t) = x'(t) &= (-2\pi\sin(2\pi t), 2\pi\cos(2\pi t)) \\
    \implies v(0) &= (0, 2\pi) \\
    v(\textstyle\frac{1}{4}) &= (-2\pi, 0) \\
    v(\textstyle\frac{1}{2}) &= (0, -2\pi) \\
    v(\textstyle\frac{3}{4}) &= (2\pi, 0) \\
    a(t) = v'(t) &= (-4\pi^2\cos(2\pi t), -4\pi^2\sin(2\pi t)) \\
    \implies a(0) &= (-4\pi^2, 0) \\
    a(\textstyle\frac{1}{4}) &= (0, -4\pi^2) \\
    a(\textstyle\frac{1}{2}) &= (4\pi^2, 0) \\
    a(\textstyle\frac{3}{4}) &= (0, 4\pi^2) \\
\end{align*}
\begin{table}[H]
    \begin{tabular}{|||c|c|c|c|c|||} \hline \hline \hline
        {\tt t} & {\tt dt} & {\bf Approximate} & {\bf Velocity} & {\bf Velocity Error}\\
         & & {\bf Velocity} & {\bf Error} & {\bf Magnitude} \\ \hline \hline \hline
        $0$ & $0.1$ & $(-1.90983, 5.87785)$ & $(1.90983, 0.4053353)$ & $1.95236967$ \\ \hline
         & $0.05$ & $(-0.978870, 6.18034)$ & $(0.978870, 0.1028453)$ & $0.9842579$ \\ \hline
         & $0.01$ & $(-0.197327, 6.27905)$ & $(0.197327, 0.0041353)$ & $0.1973703$\\ \hline \hline
        $\frac{1}{4}$ & $0.1$ & $(-5.87785, -1.90983)$ & $(0.4053353, 1.90983)$ & $1.95236967$ \\ \hline
         & $0.05$ & $(-6.18034, -0.978870)$ & $(0.1028453, 0.978870)$ & $0.9842579$ \\ \hline
         & $0.01$ & $(-6.27905, -0.197327)$ & $(0.0041353, 0.197327)$ & $0.1973703$\\ \hline \hline
        $\frac{1}{2}$ & $0.1$ & $(1.90983, -5.87785)$ & $(1.90983, 0.4053353)$ & $1.95236967$ \\ \hline
         & $0.05$ & $(0.978870, -6.180340)$ & $(0.978870, 0.1028453)$ & $0.9842579$ \\ \hline
         & $0.01$ & $(0.197327, -6.27905)$ & $(0.197327, 0.0041353)$ & $0.1973703$\\ \hline \hline
        $\frac{3}{4}$ & $0.1$ & $(5.87785, 1.90983)$ & $(0.4053353, 1.90983)$ & $1.95236967$ \\ \hline
         & $0.05$ & $(6.180340, 0.978870)$ & $(0.1028453, 0.978870)$ & $0.9842579$ \\ \hline
         & $0.01$ & $(6.27905, 0.197327)$ & $(0.0041353, 0.197327)$ & $0.1973703$\\ \hline \hline \hline
    \end{tabular}
\end{table}
\begin{table}[H]
    \begin{tabular}{|||c|c|c|c|c|||} \hline \hline \hline
        {\tt t} & {\tt dt} & {\bf Approximate} & {\bf Acceleration} & {\bf Acceleration Error}\\
         & & {\bf Acceleration} & {\bf Error} & {\bf Magnitude} \\ \hline \hline \hline
        $0$ & $0.1$ & $(-38.196601, 0.0000000)$ & $(1.281816, 0.0000000)$ & $1.281816$ \\ \hline
         & $0.05$ & $(-39.154787, 0.0000000$ & $(0.3236306, 0.0000000)$ & $0.3236306$ \\ \hline
         & $0.01$ & $(-39.465431, 0.0000000)$ & $(0.0129866, 0.0000000)$ & $0.0129866$\\ \hline \hline

        $\frac{1}{4}$ & $0.1$ & $(1.11 \times 10^{-14}, -38.196601)$ & $(1.11 \times 10^{-14}, 1.281816)$ & $1.281816$ \\ \hline
         & $0.05$ & $(0.0000000000, -39.154787$ & $(0.0000000, 0.3236306)$ & $0.3236306$ \\ \hline
         & $0.01$ & $(-1.39 \times 10^{-13}, -39.465431)$ & $(-1.39 \times 10^{-13}, 0.0129866)$ & $0.0129866$\\ \hline \hline

        $\frac{1}{2}$ & $0.1$ & $(38.196601, 0.0000000)$ & $(1.281816, 0.0000000)$ & $1.281816$ \\ \hline
         & $0.05$ & $(39.154787, -1.78 \times 10^{-13})$ & $(0.3236306, -1.78 \times 10^{-13})$ & $0.3236306$ \\ \hline
         & $0.01$ & $(39.465431, 0.0000000))$ & $(0.0129866, 0.0000000)$ & $0.0129866$\\ \hline \hline

        $\frac{3}{4}$ & $0.1$ & $(0.0000000, 38.196601)$ & $(0.0000000, 1.281816)$ & $1.281816$ \\ \hline
         & $0.05$ & $(2.22 \times 10^{-14}, 39.154787)$ & $(2.22 \times 10^{-14}, 0.3236306)$ & $0.3236306$ \\ \hline
         & $0.01$ & $(0.0000000, 39.465431)$ & $(0.0000000, 0.0129866)$ & $0.0129866$\\ \hline \hline \hline
    \end{tabular}
\end{table}
The following Python code was used to generate the above approximations:
\begin{lstlisting}[language=Python, caption=Problem 3 source code]
from __future__ import division
from math import cos, pi, exp, log

def first_order_deriv_approx(f, x, h=1e-6):
    return (f(x+h) - f(x))/h

def kin_cos(t):
    return cos(2*pi*t)

def kin_sin(t):
    return sin(2*pi*t)

def kinematics(x, y, t, dt=1.0e-6):
    x_vel = first_order_deriv_approx(x, t, dt)
    y_vel = first_order_deriv_approx(y, t, dt)
    vel = (x_vel, y_vel)

    x_acc = (x(t+dt) - 2*x(t) + x(t-dt))/(dt**2)
    y_acc = (y(t+dt) - 2*y(t) + y(t-dt))/(dt**2)
    acc = (x_acc, y_acc)

    return (vel, acc)
def problem_6():
    for t in [0, (1/4), (1/2), (3/4)]:
        print "\n    t = %.2f" % t
        for dt in [0.1, 0.05, 0.01]:
            print "\n        dt = %.2f" % dt
            (vel, acc) = kinematics(kin_cos, kin_sin, t, dt)
            print "            vel = (%.10e, %.10e)" % (vel[0], vel[1])
            print "            acc = (%.10e, %.10e)" % (acc[0], acc[1])

problem_6()
\end{lstlisting}
The following is the output of the above Python code:
\begin{Verbatim}[fontfamily=courier, numbers=left, numbersep=2pt, fontsize=\small]
    t = 0.00

        dt = 0.10
            vel = (-1.9098300563e+00, 5.8778525229e+00)
            acc = (-3.8196601125e+01, 0.0000000000e+00)

        dt = 0.05
            vel = (-9.7886967410e-01, 6.1803398875e+00)
            acc = (-3.9154786964e+01, 0.0000000000e+00)

        dt = 0.01
            vel = (-1.9732715717e-01, 6.2790519529e+00)
            acc = (-3.9465431435e+01, 0.0000000000e+00)

    t = 0.25

        dt = 0.10
            vel = (-5.8778525229e+00, -1.9098300563e+00)
            acc = (1.1102230246e-14, -3.8196601125e+01)

        dt = 0.05
            vel = (-6.1803398875e+00, -9.7886967410e-01)
            acc = (0.0000000000e+00, -3.9154786964e+01)

        dt = 0.01
            vel = (-6.2790519529e+00, -1.9732715717e-01)
            acc = (-1.3877787808e-13, -3.9465431435e+01)

    t = 0.50

        dt = 0.10
            vel = (1.9098300563e+00, -5.8778525229e+00)
            acc = (3.8196601125e+01, 0.0000000000e+00)

        dt = 0.05
            vel = (9.7886967410e-01, -6.1803398875e+00)
            acc = (3.9154786964e+01, -1.7763568394e-13)

        dt = 0.01
            vel = (1.9732715717e-01, -6.2790519529e+00)
            acc = (3.9465431435e+01, 0.0000000000e+00)

    t = 0.75

        dt = 0.10
            vel = (5.8778525229e+00, 1.9098300563e+00)
            acc = (0.0000000000e+00, 3.8196601125e+01)

        dt = 0.05
            vel = (6.1803398875e+00, 9.7886967410e-01)
            acc = (2.2204460493e-14, 3.9154786964e+01)

        dt = 0.01
            vel = (6.2790519529e+00, 1.9732715717e-01)
            acc = (0.0000000000e+00, 3.9465431435e+01)
\end{Verbatim}

\end{document}
